%!TEX root = pdl.tex

\section{Vue d’ensemble du projet}
\subsection{But du projet, portée et objectifs}
Le projet consiste à réaliser un framework complet et efficace permettant à l'entreprise cliente de fortement alléger ses délais de développement d'applications orientées réseau social.

\subsection{Hypothèses et contraintes}
Le framework devra s'intégrer dans les infrastructures matérielles et logicielles des clients, la réutilisabilité, la généricité et l'interopérabitilité de ce dernier seront indispensables.
Le frameweork devra aussi être développé sur une plateforme Java et sera aussi doté de la technologie RMI assurant la partie communication entre le client et le serveur.

\subsection{Biens livrables du projet}
Le projet sera livré sous forme de trois archives JAR : une pour la partie framework,
ainsi qu'une pour chaque application témoin (client/serveur).

\subsection{Évolution du plan de développement logiciel}
Ce Plan de Développement Logiciel sera certainement amené à être modifié lors du projet, il sera bien évidement à la disposition du client durant toute la durée du projet.
