%!TEX root = pdl.tex

\section{Introduction}
\subsection{Objectif du document}
Ce document a pour but de présenter l'ensemble des informations nécessaires au
contrôle du projet RMIAGE. Il sert à une définir une bonne gestion du projet.
Il décrit en détail les différentes phases du projet avec leur durée et les ressources nécessaires à leur bon 	déroulement.

\subsection{Portée du document}
Ce document sera utilisé par l’équipe du projet et sera disponible pour le client si celui-ci désire s'y référer.

\subsection{Définitions, acronymes et abréviations}
Dans ce document, les termes suivants seront utilisés :
\begin{itemize}
	\item Framework :
Il s'agit du produit développé par notre entreprise. Un framework est un plan de travail pour accélérer le travail de développeurs tiers.
	\item Java :
Langage de déveoppement suivant le paradigme de la programmation orientée objet. Il s'agit du langage utilisé pour développer l'application rmiage.
	\item JVM :
Java Virtual Machine, ou machine virtuelle java. Nécéssaire pour éxécuter les applications développées en Java.
	\item RMI :
Remote Method Invocation. Technologie utilisée dans le cadre de notre développement assurant
la gestion client/serveur.
\end{itemize}

\subsection{Références}
