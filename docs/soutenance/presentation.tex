














\section{Contexte}
%TODO : recuperer +/- depuis cahier des charges

\section{Objectifs}

L’objectif de ce projet était de fournir une architecture logicielle qui permette de répondre aux besoins de nos clients potentiels, ainsi qu’un cas particulier pour illustrer une partie des caractéristiques du Framework.

Notre client recevra le framwork, il pourra donc développer d’autre modules afin de le spécifier dépendamment de ces besoins.    

\section{Avant-propos}

Les applications réparties classiques sont structurées autour d'un serveur délivrant aux parties clientes un choix de services spécifiques, l'ensemble de ces services est souvent limité à l'application ou un type d'applications.

Cependant Java nous offre la technologie RMI qui offre la possibilité de communiquer entre le client et le serveur sans restriction du nombre de service a rendre, ni du type de l’application, cela veut dire que la partie serveur offre un large service pour une large variété d’application.