\documentclass[10pt,a4paper]{article}

\usepackage[utf8]{inputenc}
\usepackage[francais]{babel}



% Title Page
\title{Projet PPO : Cahier des Charges}
\author{
	Badre Baba, Geoffroy Carrier, J-Ch Saad-Dupuy,Mickael Scheer
}


\begin{document}
\maketitle

\section{Introduction}

Ce document répond à une appelle d'offre concernant la réalisation d'un framework
générique permettant la mise en place de fonctionnalités nécéssaires à la
réalisation d'applications facilitant les échanges de données dans des architectures
logicielles de type réseau social.

Les principales fonctionnalités de cet outil seront détaillées dans ce
document.

\section{Présentation du projet}
 \subsection{Contexte}
Le développement des réseaux et des moyens de communication offrent désormais beaucoup
de facilité pour communiquer et échanger du contenu de nature très diverse et ceci de manière aussi très variée. Quelque soit le contenu et la manière d’organiser ces espaces de communication où l’on partage, échange des informations, leurs objectifs sont très proches.
On le voit très nettement dans l’avènement des « réseaux sociaux » qui les déclinent sous des formes très variées.

Notre client souhaite se placer sur ce créneau des logiciels d’échange et pour cela
envisage de développer un « framework » suffisamment adaptable (système d’échange
générique) pour répondre rapidement à des diffuseurs de contenus qui voudraient proposer à leurs propres clients une zone d’échange bien spécifique. Ces clients, très exigeants en ce qui concerne l’ergonomie des progiciels qu’ils proposent nous imposent des critères de réutilisabilité, d'interopérabilité et de généricité qui seront respectées lors de la réalisation.

  \subsubsection{Intervenants}
  \subsubsection{Équipe projet}
Notre équipe se compose de 4 membres :
\begin{enumerate}
 \item Badre Baba (XXXX)
 \item Geoffroy Carrier (XXXX)
 \item Jean-Christophe Saad-Dupuy (Chef de projet)
 \item Mickael Scheer (XXXX)
\end{enumerate}

  %\subsubsection{Organisation du travail et répartition des tâches}
  \subsection{Objectifs}
L'objectif de cette réalisation est de proposer au client un outil lui permettant la réalisation de d'outils de type ``réseaux sociaux''.
Nous metterons l'accent sur sa généricité et son interopérabilité, afin de laisser à notre client la plus grande liberté dans son utilisation.

\section{Description du produit}
 \subsection{Fonctionnalités}
  \subsubsection{Serveur}
  \subsubsection{Client de démonstration}
 \subsection{Technologies utilisées}
Les préférences technologiques de notre client nous imposent le langage Java, ainsi que la techologie
Remote Method Invocation (RMI).

Dans un soucis d'interopérabilité et de liberté technologique pour le client, nous comptons utiliser le framework Hibernate en interface avec la base de données. Cet outil lui permettera une indépendance vis à vis du systeme de gestion de base de Données relationnel (SGBDR) en background.

% \subsection{Environnement de développement}
%Nous utiliserons principalement l'Environnement de Développement Intégré Eclipse, %avec le plugin Maven pour permettre l'interopérabilité avec d'autres outils, tel que %Netbeans.
%
 \subsection{Environnement de déploiement}

Notre framework se composeras d'une librairie, utilisée les applications témoins livrées.

L'utilisation de la technologie Java assure une compléte indépendance du ou des systèmes d'exploitations hôtes.

Le applications serveur et cliente pourrons être lancée depuis le même poste, en local, ou sur des machines distantes.

Les machines hôtes et serveur devrons éviement chacune disposer d'une Java Virtual Machine (JVM) afin de pouvoir executer ces applications.
\end{document}
