\documentclass[10pt,a4paper]{article}

\usepackage[utf8]{inputenc}
\usepackage[francais]{babel}



% Title Page
\title{Projet PPO : Cahier des Charges}
\author{
	Badre Baba, Geoffroy Carrier, J-Ch Saad-Dupuy,Mickael Scheer
}


\begin{document}
\maketitle

\section{Introduction}

Ce document répond à une appelle d'offre concernant la réalisaion d'un framework
générique permettant la mise en place de fonctionnalités nécéssaires à la
réalisation d'applications facilitant les échanges de données dans des architectures
logicielles de type réseau social.

Les principales fonctionnalités de cet outil seront détaillées dans ce
document.

\section{Présentation du projet}
 \subsection{Contexte}
Le développement des réseaux et des moyens de communication offrent désormais beaucoup
de facilité pour communiquer et échanger du contenu de nature très diverse et ceci de manière aussi très variée. Quelque soit le contenu et la manière d’organiser ces espaces de communication où l’on partage, échange des informations, leurs objectifs sont très proches.
On le voit très nettement dans l’avènement des « réseaux sociaux » qui les déclinent sous des formes très variées.

Notre client souhaite se placer sur ce créneau des logiciels d’échange et pour cela
envisage de développer un « framework » suffisamment adaptable (système d’échange
générique) pour répondre rapidement à des diffuseurs de contenus qui voudraient proposer à leurs propres clients une zone d’échange bien spécifique. Ces clients, très exigeants en ce qui concerne l’ergonomie des progiciels qu’ils proposent nous imposent des critères de réutilisabilité, d'interopérabilité et de généricité qui seront respectées lors de la réalisation.

  \subsubsection{Intervenants}
  \subsubsection{Équipe projet}
Notre équipe se compose de 4 membres
  %\subsubsection{Organisation du travail et répartition des tâches}
  \subsection{Objectifs}
L'objectif de cette réalisation est de proposer au client un outil lui permettant la réalisation de d'outils de type ``réseaux sociaux''.
Nous metterons l'accent sur sa généricité et son interopérabilité, afin de laisser à notre client la plus grande liberté dans son utilisation.

\section{Description du produit}
 \subsection{Fonctionnalités}
  \subsubsection{Serveur}
  \subsubsection{Client de démonstration}
 \subsection{Technologies utilisées}
Nos contraintes de réalisation nous imposent le langage Java, ainsi que la techologie
Remote Method Invocation (RMI).

Dans un soucis d'interopérabilité et de liberté pour le client, nous comptons utiliser le framework Hibernate en interface avec la base de données. Cet outil lui permettera une indépendance vis à vis du systeme de gestion de base de Données relationnel (SGBDR) en background.

% \subsection{Environnement de développement}
%Nous utiliserons principalement l'Environnement de Développement Intégré Eclipse, %avec le plugin Maven pour permettre l'interopérabilité avec d'autres outils, tel que %Netbeans.
%
 \subsection{Environnement de déploiement}

Notre framework se composeras d'une librairie, utilisée par un serveur. L'application témoin feras office de client.

L'utilisation de la technologie Java assure une compléte indépendance du ou des systèmes d'exploitations hôtes.

Le applications serveur et cliente pourrons être lancée depuis le même poste, en local, ou sur des machines distantes.

Les machines hôtes et serveur devrons éviement chacune disposer d'une Java Virtual Machine (JVM) afin de pouvoir executer ces applications.
  %\subsubsection{Logiciel}
  %\subsubsection{Matérielle}

%%NORME AFNOR
\section{Présentation Générale}
\subsection{Projet}
\subsubsection{Finalités}
\subsubsection{Espérance de retour sur investissement}

\subsection{Contexte}
\subsubsection{Situation du projet par rapport aux autres projets de l’entreprise}
\subsubsection{Études déjà effectuées}
\subsubsection{Études menées sur des sujets voisins}
\subsubsection{Suites prévues}
\subsubsection{Nature des prestations demandées}
\subsubsection{Parties concernées par le déroulement du projet et ses résultats}
% (demandeurs, utilisateurs)

\subsection{Enoncé du besoin}
%(finalités du produit pour le futur utilisateur tel que prévu par le demandeur)

\subsection{Environnement du produit recherché}
\subsubsection{Listes exhaustives des éléments} 
%(personnes, équipements, matières…) et contraintes (environnement)
\subsubsection{Caractéristiques pour chaque élément de l’environnement}

\section{Expression fonctionnelle du besoin}

\subsection{Fonctions de service et de contrainte}
\subsubsection{Fonctions de service principales}
% (qui sont la raison d’être du produit)
\subsubsection{Fonctions de service complémentaires}
% (qui améliorent, facilitent ou complètent le service rendu)
\subsubsection{Contraintes}
% (limitations à la liberté du concepteur-réalisateur)

\subsection{Critères d’appréciation}
%(en soulignant ceux qui sont déterminants pour l’évaluation des réponses)

\subsection{Niveaux des critères d’appréciation et ce qui les caractérise}
\subsubsection{Niveaux dont l’obtention est imposée}
\subsubsection{Niveaux souhaités mais révisables}

\subsection{Cadre de réponse}

\subsubsection{Pour chaque fonction}
\begin{enumerate}
\item{Solution proposée}
\item{Niveau atteint pour chaque critère d’appréciation de cette fonction et modalités de contrôle}
\item{Part du prix attribué à chaque fonction}
\end{enumerate}

\subsubsection{Pour l’ensemble du produit}
\begin{enumerate}
\item{Prix de la réalisation de la version de base}
\item{Options et variantes proposées non retenues au cahier des charges}
\item{Mesures prises pour respecter les contraintes et leurs conséquences économiques}
\item{Outils d’installation, de maintenance … à prévoir}
\item{Décomposition en modules, sous-ensembles}
\item{Prévisions de fiabilité}
\item{ Perspectives d’évolution technologique}
\end{enumerate}

\end{document}
