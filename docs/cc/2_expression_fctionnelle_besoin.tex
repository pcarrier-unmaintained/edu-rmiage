%!TEX root = cahier_charges_afnor.tex

\section{Expression fonctionnelle du besoin}

\subsection{Fonctions de service et contraintes}

\subsubsection{Fonctions de service principales}

\emph{Framework} accélérant le développement de serveurs applicatifs et clients lourds et proposant~:

\begin{enumerate}
	\item La gestion d'utilisateurs~;
	\item Le stockage de contenus~;
	\item La création, modification, suppression de ces contenus~;
	\item Le partage de ces contenus entre utilisateurs~;
	\item Un accès aisé à ces contenus~;
	\item Une architecture pour intégrer une politique de permissions~;
	\item Des outils de communication entre les utilisateurs.
\end{enumerate}

\subsubsection{Fonctions de service complémentaires}

Deux programmes de démonstration exploitant les fonctionnalités du \emph{framework} seront fournies~:

\begin{itemize}
 \item Un programme serveur proposant un système de forums~;
 \item Un programme client exposant les fonctionnalités du serveur.

\end{itemize}

De plus, nous fournirons une batterie de tests unitaires garantissant la qualité de chaque fonctionnalité du \emph{framework}.

\subsubsection{Contraintes}
% (limitations à la liberté du concepteur-réalisateur)
\begin{enumerate}

	\item Développement

Le framework sera aussi indépendant de choix techniques que possible, notamment au niveau persistance des données qui sera totalement générique.

	\item Environnement

L'utilisation de la technologie Java assure une complète indépendance du ou des systèmes d'exploitations hôtes.

Les applications serveur et cliente pourront être lancée depuis le même poste, en réseau local ou \emph{via} Internet.

Les machines hôtes et serveur devront chacune disposer d'une Java Virtual Machine (JVM) afin de pouvoir exécuter ces applications.

\end{enumerate}

\subsection{Critères d’appréciation}

Les principaux critères d'appréciation sont~:

\begin{itemize}
	\item Du coté \emph{framework} :
	\begin{itemize} 
		\item Sa qualité ;
		\item Son indépendance concernant le système de persistance ;
		\item Sa flexibilité ;
		\item Sa généricité ;
	\end{itemize} 
	\item du coté application témoin :
	\begin{itemize}
		\item Leur pertinence vis-à-vis des fonctionnalités principales du \emph{framework} ;
		\item La facilité de leur utilisation pour des tests.
	\end{itemize}
\end{itemize}


%\subsection{Niveaux des critères d’appréciation et ce qui les caractérise}
%\subsubsection{Niveaux dont l’obtention est imposée}
%\subsubsection{Niveaux souhaités mais révisables}
