\section{Expression fonctionnelle du besoin}

\subsection{Fonctions de service et contrainte}
\subsubsection{Fonctions de service principales}
% (qui sont la raison d’être du produit)
Le framework permettra de créer et d'utiliser différents type d'objets tel que :

\begin{itemize}
\item{des utilisateurs avec différents attributs (utilisateur classique, administrateur)}
\item{des groupes}
\item{des fichiers}
\item{des messages}
\end{itemize}

\subsubsection{Fonctions de service complémentaires}
% (qui améliorent, facilitent ou complètent le service rendu)
Deux programmes de démonstration exploitant les fonctionnalités du framework serons également
fournis:
\begin{itemize}
 \item un programme serveur, basé sur notre produit ainsi que la technologie RMI, mettant à disposition des ressources ;
 \item un programme client, exploitant les ressources fournies par le serveur. 
\end{itemize}

De plus, nous fournirons une batterie de tests unitaire garantissant la qualité de chaque fonctionnalité du framework.
\subsubsection{Contraintes}
% (limitations à la liberté du concepteur-réalisateur)
\begin{enumerate}
 \item Développement

Le framework doit être au maximum indépendant, notemment au niveau de l'acces au données.
Le client doit avoir la possibilité d'utiliser le systeme de gestion de base de données relationnelle (SGBDR) de son choix.

 \item Environnement

Notre framework se composera d'une librairie, utilisée par les applications témoins livrées.

L'utilisation de la technologie Java assure une complète indépendance du ou des systèmes d'exploitations hôtes.

Les applications serveur et cliente pourrons être lancée depuis le même poste, en local, ou sur des machines distantes.

Les machines hôtes et serveur devrons chacune disposer d'une Java Virtual Machine (JVM) afin de pouvoir executer ces applications.

\end{enumerate}

\subsection{Critères d’appréciation}
%(en soulignant ceux qui sont déterminants pour l’évaluation des réponses)
%TODO revoir un peu les critères

Les principaux critères d'appréciation seront : 

\begin{itemize}
 \item du coté Framework :
 \begin{itemize} 
  \item son indépendance
  \item sa qualité
  \item sa flexibilité
 \end{itemize} 
 
 \item du coté applications
 \begin{itemize}
  \item leur pertinence
  \item leur facilité de mise en place
 \end{itemize}
\end{itemize}


\subsection{Niveaux des critères d’appréciation et ce qui les caractérise}
\subsubsection{Niveaux dont l’obtention est imposée}
\subsubsection{Niveaux souhaités mais révisables}
