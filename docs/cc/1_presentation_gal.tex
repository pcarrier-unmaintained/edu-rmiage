\section{Présentation Générale}
\subsection{Projet}
\subsubsection{Finalités}

La réalisation a effectuer par notre équipe porte sur le développement logiciel d'une bibliothèque (framework) permettant des échanges de données au sein de structures logicielles développées par le client, de type ``réseau social`` 

%\subsubsection{Espérance de retour sur investissement}

\subsection{Contexte}

Le développement des réseaux et des moyens de communication offrent désormais beaucoup
de facilité pour communiquer et échanger du contenu de nature très diverse et ceci de manière aussi très variée. Quelque soit le contenu et la manière d’organiser ces espaces de communication où l’on partage, échange des informations, leurs objectifs sont très proches.
On le voit très nettement dans l’avènement des « réseaux sociaux » qui les déclinent sous des formes très variées.

Notre client souhaite se placer sur ce créneau des logiciels d’échange et pour cela
envisage de développer un « framework » suffisamment adaptable (système d’échange
générique) pour répondre rapidement à des diffuseurs de contenus qui voudraient proposer à leurs propres clients une zone d’échange bien spécifique. Ces clients, très exigeants en ce qui concerne l’ergonomie des progiciels qu’ils proposent nous imposent des critères de réutilisabilité, d'interopérabilité et de généricité qui seront respectées lors de la réalisation.

\subsubsection{Situation du projet par rapport aux autres projets de l’entreprise}

Bien que notre équipe soit compétente et performante en développement, notemment dans les technologies Java, il s'agit ici de notre première réalisation commerciale.

\subsubsection{Études déjà effectuées}
% TODO : Bluff ?
\subsubsection{Études menées sur des sujets voisins}
% TODO : Bluff ?
\subsubsection{Suites prévues}
\subsubsection{Nature des prestations demandées}

% Heu...cf Finalités....assez identique finalement.

\subsubsection{Parties concernées par le déroulement du projet et ses résultats}
% (demandeurs, utilisateurs)

Durant toute la durée du projet, nous serons en contact avec le représentant du comité de
pilotage, M. François Puitg.

Les utilisateurs finaux seront les clients utilisateurs des produits développés par XXX,
diffuseurs de contenus, auprès de qui nous n'interviendrons pas.

\subsection{Enoncé du besoin}
%(finalités du produit pour le futur utilisateur tel que prévu par le demandeur)
Les clients de XXX, difuseurs de contenus, souhaitent proposer à leurs propres clients des
zones d'échanges bien spécifiques et nécéssitent une architecture logicielle réutilisable leur mettant à disposition les fonctionalités de bases pouvant être spécialisées suivant leurs besoins.

\subsection{Environnement du produit recherché}

Le framework doit être réalisé en Java, en utilisant la technologie Remote Method Invocation (RMI).

\subsubsection{Listes exhaustives des éléments} 
%(personnes, équipements, matières…) et contraintes (environnement)
HA....HA....HA
% TODO trouver la liste exhaustive des éléments
\subsubsection{Caractéristiques pour chaque élément de l’environnement}
